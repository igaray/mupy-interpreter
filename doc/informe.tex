
\documentclass [a4paper,titlepage]{report}
%\pagestyle{plain}
\usepackage{makeidx}
\usepackage[spanish]{babel}
\usepackage[utf8x]{inputenc}
\usepackage[T1]{fontenc}
\usepackage{lmodern}
\usepackage{enumerate}
\usepackage{longtable}
\usepackage{graphicx}
\usepackage{ucs}
\usepackage{textcomp}
\usepackage{alltt}
\usepackage{listings}
\usepackage{moreverb}
\usepackage{upquote}

\setlength{\parskip}{1ex plus 0.5ex minus 0.2ex}

\title{Proyecto 2\\Compiladores e Intérpretes 2012\\Intérprete $\mu$Py}

\author{Garay, I\~{n}aki LU 67387}

\makeindex

\begin{document}

\maketitle

\tableofcontents

\chapter{Uso e Invocación}

\chapter{Componentes léxicos}

\begin{itemize}
\item Identificadores
\item Numeros
\item Strings
\item Punto
\item Palabra clave \texttt{print}
\item Operadores de asignacion
\item Operadores aritmeticos
\item Operadores de slice
\end{itemize}

La siguiente tabla muestra los tokens reconocidos por el analizador léxico.

\begin{longtable}{l | l | l}
\bfseries{Token}     & \bfseries{Expresión Regular}                                          & \bfseries{Ejemplo}                \tabularnewline \endhead
TK\_IDENTIFIER       & [a-zA-Z\_{}\textbackslash{}\${}][a-zA-Z\_{}\textbackslash{}\${}0-9]*  & identifier                        \tabularnewline
TK\_INTEGER          & (0\textbar{}[1-9]([0-9])*)                                            & 42                                \tabularnewline
TK\_STRING           & (\textbackslash{}"\textbackslash{}"\textbar{}\textbackslash{}"([\^{}(\textbackslash{}"\textbackslash{}\textbackslash{})]\textbar{}[\textbackslash{}\textbackslash{}\textbackslash{}\textbackslash{}\textbar{}\textbackslash{}\textbackslash{}'\textbar{}\textbackslash{}\textbackslash{}\textbackslash{}"\textbar{}\textbackslash{}\textbackslash{}n])*\textbackslash{}") & \textquotedbl{}s\textquotedbl{}   \tabularnewline
TK\_BRACKET\_OPEN    & \textbackslash{}[                                                     & [                                 \tabularnewline
TK\_BRACKET\_CLOSE   & \textbackslash{}]                                                     & ]                                 \tabularnewline
TK\_PAREN\_OPEN      & \textbackslash{})                                                     & (                                 \tabularnewline
TK\_PAREN\_CLOSE     & \textbackslash{}(                                                     & )                                 \tabularnewline
TK\_PERIOD           & .                                                                     & .                                 \tabularnewline
TK\_ASSIGNMENT       & =                                                                     & =                                 \tabularnewline
TK\_ASSIGNMENT\_ADD  & +=                                                                    & =                                 \tabularnewline
TK\_ASSIGNMENT\_MUL  & *=                                                                    & =                                 \tabularnewline
TK\_ADD              & +                                                                     & +                                 \tabularnewline
TK\_SUB              & -                                                                     & -                                 \tabularnewline
TK\_MUL              & *                                                                     & *                                 \tabularnewline
TK\_DIV              & \textbackslash{}/                                                     & /                                 \tabularnewline
TK\_PRINT            & print                                                                 & print                             \tabularnewline
\end{longtable}

\chapter{Gramática}

\begin{verbatim}
<program>           := <statement_list> .

<statement_list>    := <statement_list> enter <statement>
                     | <statement>

<statement>         := <assignment>
                     | <print>

<print>             := print <expression>

<assignment>        := id  = <expression>
                     | id += <expression>
                     | id *= <expression>

<expression>        := <expression> + <expression>
                     | <expression> - <expression>
                     | <expression> * <expression>
                     | <expression> / <expression>
                     | id
                     | id <slice>
                     | num
                     | string
                     | string <slice>
                     | str <expression>
                     | ( <expression> )

<slice>             := [ <expression> ]
                     | [ <expression> : <expression> ]
\end{verbatim}

\chapter{Definición Dirigida por la Sintaxis}

\chapter{Detalles de implementación}

\end{document}
